\documentclass[a4paper,12pt]{article}
\usepackage[hmargin=2cm,vmargin=1cm,bmargin=2cm]{geometry}
\usepackage[utf8]{inputenc}
\usepackage[brazil]{babel}
\usepackage{framed}
\usepackage{caption}
\usepackage{xcolor}
\usepackage{textcomp}

% Setting new colors
\definecolor{verde}{rgb}{0,0.5,0}
\definecolor{lbcolor}{rgb}{0.9,0.9,0.9}
\captionsetup[lstlisting]{singlelinecheck=false, margin=0pt, font={bf,footnotesize}}

% Setting layout to display C++ code
\usepackage{listingsutf8}

\lstset{
    backgroundcolor=\color{lbcolor},
    captionpos=b,
    tabsize=4,
    language=[GNU]C++,
    upquote=true,
    aboveskip={1.5\baselineskip},
    columns=fixed,
    showstringspaces=false,
    breaklines=true,
    frameround=fttt,
    prebreak = \raisebox{0ex}[0ex][0ex]{\ensuremath{\hookleftarrow}},
    showtabs=false,
    showspaces=false,
    showstringspaces=false,
    identifierstyle=\ttfamily,
	frame=lines,
	numbers=left,
	numberstyle=\tiny,
	numbersep=5pt,
	breaklines=true,
	basicstyle=\footnotesize,
	keywordstyle=\color[rgb]{0,0,1}\bfseries,
	commentstyle=\color{verde},
	stringstyle=\color{red}
}


%title and author details
\title{TRABALHO DA UNIDADE 1 - Análise de Algoritmos}
\author{
\textbf{Disciplina:} IMD0032 - EDB II
\\\textbf{Semestre:} 2014/I
\\\textbf{Professor:} Carlos A. Prolo
\\\textbf{Alunos:} José Bernardo Gurgel, Thiago César
}

\date{} %remove date

\begin{document}
\maketitle
%First question (Selection Sort)
\section{Questão}
\begin{description}
    \item{\textbf{a)}} Nesta questão, as instruções mais executadas são a da linha 26, a condição da linha 25, o k++ da linha 24 e o $k < n$ da linha 24. Mesmo que o número de vezes de execução dessas linhas sejam diferentes ($k < n$ é a mais executada), assintoticamente é a mesma coisa. O número de vezes que $k < n$ é executada é:
	$$
	\sum_{j=1}^{n-1} \sum_{k=j+1}^{n-1} 1
	$$
No entanto, não é necessário resolvermos exatamente estes somatórios para concluir que, tanto no melhor caso quanto no pior caso, o resultado é $\theta(n^2)$. Nesse caso, em que há dois \textit{for} embutidos, é uma fração quadrática $\theta(n^2)$.

	\item{\textbf{b)}} Complete usando notação assintótica: O número de comparações feitas pelo algoritmo no pior caso é \underline{$\theta(n^2)$}. No entanto o número de SWAPS no pior caso é apenas \underline{$\theta(n)$}.
	\item{\textbf{c)}} No segundo algoritmo, é utilizada a função \texttt{menor}, no qual recebe como um dos parâmetros um vetor \texttt{v}. O que temos que observar é que este vetor \textbf{não} é passado como uma referência, como um ponteiro. E isto é muito relevante para a performance do algoritmo. O que está acontecendo é que quando passamos o vetor como parâmetro, a função irá criar um \textbf{novo} \texttt{array} e percorrer o que foi passado como parâmetro para poder criar uma cópia no novo. Ou seja, apesar de não ser um problema explícito, o algotimo acaba criando novos percursos que afetam sua performance.
\end{description}

%Second question (Mergesort)
\section{Questão}
\begin{description}
    \item{\textbf{a)}}
	\item{\textbf{b)}}
	\item{\textbf{c)}}
\end{description}

%Third question (Análise do Quicksort)
\section{Questão}
\begin{description}
    \item{\textbf{a)}}
	\item{\textbf{b)}}
	\item{\textbf{c)}}
	\item{\textbf{d)}}
\end{description}

%Fourth question (Análise de pior caso da inserção de um elemento na Hash)
\section{Questão}
Entre as muitas aplicações do \textit{hash}, referimos a implementação eficiente dos métodos de tabelação. Estes métodos são usados em pesquisas heurísticas, e em jogos, por exemplo, para guardar o valor de configurações de um tabuleiro de xadrez.
Quando inserimos um valor $x$ e posteriormente um valor $y$ com $h(y)$ = $h(x)$, temos uma \textit{colisão}. A posição da tabela para onde $y$ deveria ir já está ocupada e terá que existir um método para resolver as colisões.

A probabilidade $p$ de se inserir $N$ itens consecutivos sem colisão em uma tabela de tamanho $M$ é:
	$$
	p = \frac{M-1}{M}\times\frac{M-2}{M}\times...\times\frac{M-N+1}{M} = \prod \limits_{i=1}^N \frac{M-i+1}{M} = \frac{M!}{(M-N)!M^N}
	$$

Assim, uma das formas de resolver as \textit{colisões} é simplesmente construir uma lista linear encadeada para cada endereço da tabela, desse modo, todas as chaves com o mesmo endereço são encadeadas em uma lista linear.

A função de inserção, nesse caso, \textit{CHAINED-HASH-INSERT(T, x)} insere o elemento $x$ na cabeça da lista $T[h(x.key)]$. Portanto, o tempo de execução no pior caso, para a operação de inserção, é $\theta(1)$.

%Fifth question (Construção de um algorítmo)
\section{Questão}
\lstinputlisting[caption=Programa que exibe uma mensagem se a soma dos valores de um conjunto A é igual à soma dos valores em S-A., language=C++]{is_subset_sum.cpp}

Suponhamos que temos $n = 500$ números, no entanto, sabemos que a soma de todos os números é no máximo $N = 10000$. Esse pequeno detalhe faz com que o problema possa ser resolvido com tempo de execução igual a $\theta(n\times N)$.

\end{document}
