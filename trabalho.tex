\documentclass[a4paper,12pt]{article}
\usepackage[hmargin=2cm,vmargin=1cm,bmargin=2cm]{geometry}
\usepackage[utf8]{inputenc}
\usepackage[brazil]{babel}
\usepackage{framed}
\usepackage{caption}
\usepackage{xcolor}
\usepackage{textcomp}
\usepackage{fancybox}
\usepackage{tikz}

% Setting new colors
\definecolor{verde}{rgb}{0,0.5,0}
\definecolor{lbcolor}{rgb}{0.9,0.9,0.9}

% Setting caption
\captionsetup[lstlisting]{singlelinecheck=false, margin=0pt, font={bf,footnotesize}}

% Setting layout to display C++ code
\usepackage{listingsutf8}

\lstset{
    backgroundcolor=\color{lbcolor},
    captionpos=b,
    tabsize=4,
    language=[GNU]C++,
    upquote=true,
    aboveskip={1.5\baselineskip},
    columns=fixed,
    showstringspaces=false,
    breaklines=true,
    frameround=fttt,
    prebreak = \raisebox{0ex}[0ex][0ex]{\ensuremath{\hookleftarrow}},
    showtabs=false,
    showspaces=false,
    showstringspaces=false,
    identifierstyle=\ttfamily,
	frame=lines,
	numbers=left,
	numberstyle=\tiny,
	numbersep=5pt,
	breaklines=true,
	basicstyle=\footnotesize,
	keywordstyle=\color[rgb]{0,0,1}\bfseries,
	commentstyle=\color{verde},
	stringstyle=\color{red}
}


%title and author details
\title{TRABALHO DA UNIDADE 1 - Análise de Algoritmos}
\author{
\textbf{Disciplina:} IMD0032 - EDB II
\\\textbf{Semestre:} 2014/I
\\\textbf{Professor:} Carlos A. Prolo
\\\textbf{Alunos:} José Bernardo Gurgel, Thiago César
}

\date{} %remove date

\begin{document}
	
\maketitle
%First question (Selection Sort)
\section{Questão}
\begin{description}
    \item{\textbf{a)}} Nesta questão, as instruções mais executadas são a da linha 26, a condição da linha 25, o k++ da linha 24 e o $k < n$ da linha 24. Mesmo que o número de vezes de execução dessas linhas sejam diferentes ($k < n$ é a mais executada), assintoticamente é a mesma coisa. O número de vezes que $k < n$ é executada é:
	$$
	\sum_{j=1}^{n-1} \sum_{k=j+1}^{n-1} 1
	$$
No entanto, não é necessário resolvermos exatamente estes somatórios para concluir que, tanto no melhor caso quanto no pior caso, o resultado é $\Theta(n^2)$. Nesse caso, em que há dois \textit{for} embutidos, é uma fração quadrática $\Theta(n^2)$.

	\item{\textbf{b)}} Complete usando notação assintótica: O número de comparações feitas pelo algoritmo no pior caso é \underline{$\Theta(n^2)$}. No entanto o número de SWAPS no pior caso é apenas \underline{$\Theta(n)$}.
	
	\item{\textbf{c)}} No segundo algoritmo, é utilizada a função \texttt{menor}, no qual recebe como um dos parâmetros um vetor \texttt{v}. O que temos que observar é que este vetor \textbf{não} é passado como uma referência, como um ponteiro. E isto é muito relevante para a performance do algoritmo. O que está acontecendo é que quando passamos o vetor como parâmetro, a função irá criar um \textbf{novo} \texttt{array} e percorrer o que foi passado como parâmetro para poder criar uma cópia no novo. Ou seja, apesar de não ser um problema explícito, o algoritmo acaba criando novos percursos que afetam sua performance.
\end{description}

%Second question (Mergesort)
\section{Questão}
\begin{description}
    \item{\textbf{a)}} Demonstração da análise do algoritmo \textit{mergesort} com divisão feita em partes de tamanhos complementares $\lfloor n/4 \rfloor$ e $\lceil 3n/4 \rceil$, utilizando o método da árvore de recursão, percebemos que a execução no pior caso do algoritmo, ainda permanece como $\Theta (n log(n))$:

	\ovalbox{
	\begin{tikzpicture}[level/.style={sibling distance=20mm/#1}]
	\node [] (z){$n$}
	  child {node [] (a) {$\frac{n}{4}$}
	    child {node [] (b) {$\frac{n}{4^2}$}
	      child {node {$\vdots$}
	        child {node [] (d) {$\frac{n}{4^k}$}}
	        child {node [] (d2) {$\frac{n}{4^k}$}}
	        child {node [] (d3) {$\frac{n}{4^k}$}}
	        child {node [] (d4) {$\frac{n}{4^k}$}}
	      } 
	      child {node {$\vdots$}}
	      child {node {$\vdots$}}
	      child {node {$\vdots$}}
	    }
	    child {node [] (b2) {$\frac{n}{4^2}$}}
	    child {node [] (b3) {$\frac{n}{4^2}$}}
	    child {node [] (b4) {$\frac{n}{4^2}$}
	      child {node {$\vdots$}}
	      child {node {$\vdots$}}
	      child {node {$\vdots$}}
	      child {node {$\vdots$}}
	    }
	  }
	  child {node [] (a2) {$\frac{n}{4}$}}
	  child {node [] (a3) {$\frac{n}{4}$}}
	  child {node [] (a4) {$\frac{n}{4}$}
	    child {node [] (b5) {$\frac{n}{4^2}$}
	      child {node {$\vdots$}}
	      child {node {$\vdots$}}
	      child {node {$\vdots$}}
	      child {node {$\vdots$}}
	    }
	    child {node [] (b6) {$\frac{n}{4^2}$}}
	    child {node [] (b7) {$\frac{n}{4^2}$}}
	    child {node [] (b8) {$\frac{n}{4^2}$}
	      child {node {$\vdots$}}
	      child {node {$\vdots$}}
	      child {node {$\vdots$}}
	      child {node (c){$\vdots$}
	        child {node [] (o) {$\frac{n}{4^k}$}}
	        child {node [] (o2) {$\frac{n}{4^k}$}}
	        child {node [] (o3) {$\frac{n}{4^k}$}}
	        child {node [] (o4) {$\frac{n}{4^k}$}
	          child [grow=right] {node (q) {$=$} edge from parent[draw=none]
	            child [grow=right] {node (q) {$\Theta_{k = \lg n}(n)$} edge from parent[draw=none]
	              child [grow=up] {node (r) {$\vdots$} edge from parent[draw=none]
	                child [grow=up] {node (s) {$\Theta_2(n)$} edge from parent[draw=none]
	                  child [grow=up] {node (t) {$\Theta_1(n)$} edge from parent[draw=none]
	                    child [grow=up] {node (u) {$\Theta_0(n)$} edge from parent[draw=none]}
	                  }
	                }
	              }
	              child [grow=down] {node (v) {$\Theta(n \cdot \lg n)$}edge from parent[draw=none]}
	            }
	          }
	        }
	      }
	    }
	  };
	\path (a) -- (a2) node [midway] {+};
	\path (a2) -- (a3) node [midway] {+};
	\path (a3) -- (a4) node [midway] {+};
	\path (b) -- (b2) node [midway] {+};
	\path (b2) -- (b3) node [midway] {+};
	\path (b3) -- (b4) node [midway] {+};
	\path (b4) -- (b5) node [midway] {+};
	\path (b5) -- (b6) node [midway] {+};
	\path (b6) -- (b7) node [midway] {+};
	\path (b7) -- (b8) node [midway] {+};
	\path (d) -- (d2) node [midway] {+};
	\path (d2) -- (d3) node [midway] {+};
	\path (d3) -- (d4) node [midway] {+};
	\path (o) -- (o2) node [midway] {+};
	\path (o2) -- (o3) node [midway] {+};
	\path (o3) -- (o4) node [midway] {+};
	\path (o) -- (d4) node (x) [midway] {$\cdots$}
	  child [grow=down] {
	    node (y) {$\Theta\left(\displaystyle\sum_{i = 0}^k 4^i \cdot \frac{n}{4^i}\right)$}
	    edge from parent[draw=none]
	  };
	\path (q) -- (r) node [midway] {+};
	\path (s) -- (r) node [midway] {+};
	\path (s) -- (t) node [midway] {+};
	\path (s) -- (b8) node [midway] {=};
	\path (t) -- (u) node [midway] {+};
	\path (z) -- (u) node [midway] {=};
	\path (a4) -- (t) node [midway] {=};
	\path (y) -- (x) node [midway] {$\Downarrow$};
	\path (v) -- (y)
	  node (w) [midway] {$\Theta\left(\displaystyle\sum_{i = 0}^k n\right) = \Theta(k \cdot n)$};
	\path (q) -- (v) node [midway] {=};
	\path (d4) -- (x) node [midway] {+};
	\path (o4) -- (x) node [midway] {+};
	\path (y) -- (w) node [midway] {$=$};
	\path (v) -- (w) node [midway] {$\Leftrightarrow$};
	\path (r) -- (c) node [midway] {$\cdots$};
	\end{tikzpicture}}

	\item{\textbf{b)}} De acordo com a questão anterior, generalizando a expressão $\lfloor n/k \rfloor$ e $\lceil ((k-1)/k)n \rceil$, para um inteiro $k\geq 2$ podemos demonstrar da seguinte forma:
	
	$$T(n)=\left\{\begin{array}{rc}
	1,&\mbox{se}\quad n = 1,\\
	kT(n/k)+n, &\mbox{se}\quad n > 1.
	\end{array}\right.
	$$
	
	Aplicando o teorema master $T(n) = aT(n/b)+f(n)$ podemos verificar que: 
	\begin{itemize}
		\item $f(n) = \Theta(n^{\log_k k}) = \Theta(n)$, logo, $T(n) = \Theta(nlog(n))$
	\end{itemize}
	
	Concluindo, para todos os casos em que $\lfloor n/k \rfloor$ e $\lceil ((k-1)/k)n \rceil$, a complexidade do algoritmo no pior caso será de $\Theta (n log(n))$ 

	\item{\textbf{c)}} Se $k=1$ então, $\lfloor n/1 \rfloor$ e $\lceil ((1-1)/1)n \rceil$, simplificando $T(n)+n$, logo:
	\begin{itemize}
		\item $f(n) = \Theta(n^{\log_1 1}) = \Theta(n)$, logo, $T(n) = \Theta(nlog(n))$
	\end{itemize}
\end{description}

%Third question (Análise do Quicksort)
\section{Questão}
\begin{description}
    \item{\textbf{a)}}
	\item{\textbf{b)}}
	\item{\textbf{c)}}
	\item{\textbf{d)}}
\end{description}

%Fourth question (Análise de pior caso da inserção de um elemento na Hash)
\section{Questão}
Entre as muitas aplicações do \textit{hash}, referimos a implementação eficiente dos métodos de tabelação. Estes métodos são usados em pesquisas heurísticas, e em jogos, por exemplo, para guardar o valor de configurações de um tabuleiro de xadrez.
Quando inserimos um valor $x$ e posteriormente um valor $y$ com $h(y)$ = $h(x)$, temos uma \textit{colisão}. A posição da tabela para onde $y$ deveria ir já está ocupada e terá que existir um método para resolver as colisões.

A probabilidade $p$ de se inserir $N$ itens consecutivos sem colisão em uma tabela de tamanho $M$ é:
	$$
	p = \frac{M-1}{M}\times\frac{M-2}{M}\times...\times\frac{M-N+1}{M} = \prod \limits_{i=1}^N \frac{M-i+1}{M} = \frac{M!}{(M-N)!M^N}
	$$

Assim, uma das formas de resolver as \textit{colisões} é simplesmente construir uma lista linear encadeada para cada endereço da tabela, desse modo, todas as chaves com o mesmo endereço são encadeadas em uma lista linear.

A função de inserção, nesse caso, \textit{CHAINED-HASH-INSERT(T, x)} insere o elemento $x$ na cabeça da lista $T[h(x.key)]$. Portanto, o tempo de execução no pior caso, para a operação de inserção, é $\Theta(1)$.

%Fifth question (Construção de um algorítmo)
\section{Questão}
\lstinputlisting[caption=Programa que exibe uma mensagem se a soma dos valores de um conjunto A é igual à soma dos valores em S-A., language=C++]{is_subset_sum.cpp}

Suponhamos que temos $n = 500$ números, no entanto, sabemos que a soma de todos os números é no máximo $N = 10000$. Esse pequeno detalhe faz com que o problema possa ser resolvido com tempo de execução igual a $\Theta(n\times N)$.

\end{document}
